\HeaderA{RpadGraphing}{Rpad graphing utilities}{RpadGraphing}
\aliasA{fspdf}{RpadGraphing}{fspdf}
\aliasA{graphoptions}{RpadGraphing}{graphoptions}
\aliasA{newgraph}{RpadGraphing}{newgraph}
\aliasA{RpadPlotName}{RpadGraphing}{RpadPlotName}
\aliasA{showgraph}{RpadGraphing}{showgraph}
\keyword{math}{RpadGraphing}
\begin{Description}\relax
Utilities to make graphing in Rpad R scripts easier.
\end{Description}
\begin{Usage}
\begin{verbatim}
  fspdf(file = "Rplot.pdf", width = 4, height = 4, ...)
  graphoptions(..., reset = FALSE, override.check = FALSE)
  newgraph(name = "", type = graphoptions()$type, res = graphoptions()$res,
     width = graphoptions()$width, height = graphoptions()$height,
     pointsize = graphoptions()$pointsize, sublines = graphoptions()$sublines,
     toplines = graphoptions()$toplines, ratio = graphoptions()$ratio,
     leftlines = graphoptions()$leftlines, lwd = graphoptions()$lwd, ...)
  showgraph(name = RpadPlotName(), link = FALSE, ...)
  RpadPlotName()
\end{verbatim}
\end{Usage}
\begin{Arguments}
\begin{ldescription}
\item[\code{file}] is the file name of the pdf file.
\item[\code{reset}] == To define == 
\item[\code{override.check}] == To define == 
\item[\code{name}] is the name of the graph with the extension ('.png') left OFF.
\item[\code{type}] is the ghostscript device. It
could be "png256", "pngalpha" (the default), "pdf", or any other
postscript device.
\item[\code{res}] is the resolution of the bitmap in dots per inch.
\item[\code{width, height}] are the dimensions of the graph in inches.
\item[\code{ratio}] specifies the ratio of the graph if either the width or
height is not specified.
\item[\code{pointsize}] is the font point size passed to the postscript
device.
\item[\code{sublines, toplines, leftlines}] specify the dimension of
the graph along with the outside border. It defaults to fairly tight
outside dimensions.
\item[\code{lwd}] is the line width set with \code{par}.
\item[\code{link}] is a logical specifying whether to show a link to allow
the user to download an EPS file of the graph (not available when
using R's png driver.
\item[\code{...}] in \code{fspdf} and \code{newgraph}, arguments are passed
to \code{postscript}. In \code{graphoptions}, the arguments are
assigned as defaults for \code{newgraph}.
\end{ldescription}
\end{Arguments}
\begin{Details}\relax
\code{fspdf} is a device driver that generates a PDF file that opens in
full-screen mode. The advantage of this is that it doesn't have all of
the toolbars and other junk, so when a PDF is embedded in a page, it
looks nice and fits nicely. Internally, it uses the postscript device
along with ghostscript to generate the PDF. Note that many users do not
have the PDF plug-in set up to display inline graphics.

The \code{graphoptions}, \code{newgraph}, and \code{showgraph} set of functions
allows quick setup and display of web-friendly graphics. The user can
normally just use any of the plot commands followed by
\code{showgraph}. \code{newgraph} sets up the graphics
device, and it's called when the Rpad package
starts. \code{showgraph} generates the HTML to show the graph and runs
\code{newgraph} to advance
to the next graphics file. The user only runs \code{newgraph} to change
parameters from their defaults. Graphics files are by default named
Rpad\_plot1, Rpad\_plot2, and so on. Named graphs can also be used, but
there's more of a chance that if the user has caching set wrong (or the
server's caching is set wrong) that graphs won't update properly in the
user's browser. With the default sequential numbering of files, caching
problems are less likely. \code{graphoptions} is also available to change the
defaults for subsequent graphs.

Internally, \code{newgraph} uses the postscript device and ghostscript
to generate the bitmap for the browser. The pngalpha device of
ghostscript does
anti-aliasing for smoother-looking PNG output. Also, this approach has the
side-effect of creating an EPS file for each graph, so it's easy to add
a link to allow the user to download the EPS file. Another approach is to
use the \code{png} device directly (but on linux, this requires an X
server, and results are not antialiased).
\end{Details}
\begin{Value}
\code{RpadPlotName()} returns the name of the currently active
plot. None of the other routines return a value: all are run for their
side effects.
\end{Value}
\begin{Author}\relax
Tom Short, EPRI Solutions, Inc., (\email{tshort@eprisolutions.com})
\end{Author}
\begin{SeeAlso}\relax
See also \code{bitmap}, \code{png}, and \code{pdf}.
\end{SeeAlso}
\begin{Examples}
\begin{ExampleCode}
# make some graphs (a default graphics device is already available)
  x <- 1:10
  y <- x^2
  y2 <- x^3
  if (capabilities("png")) graphoptions(type="Rpng")
  newgraph()
  plot(x, y)  # does the plot
  plot(x, y2) # does the second plot
  HTMLon()    # sets Rpad to HTML output
  showgraph() # closes the device, outputs the HTML for the both
              # images, and creates the next device
  plot(x, y2)
  showgraph()

# graphs with named files:
  newgraph("graph_A")
  plot(x, y)
  showgraph("graph_A")
  newgraph("graph_B", width = 4, height = 6) # also adjust the width and height
  plot(x, y2)
  showgraph("graph_B")
\end{ExampleCode}
\end{Examples}

